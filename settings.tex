% !TEX encoding = UTF-8 Unicode
% !TEX TS-program = LuaLaTeX
% !TEX root = memoire.tex
% !TEX spellcheck = fr-FR

%*********************************************************************************
% Preamble
%*********************************************************************************

%*********************************************************************************
% TOC settings
%*********************************************************************************
% \setcounter{secnumdepth}{5}
% \setcounter{tocdepth}{5}

%*********************************************************************************
% Chapters style
%*********************************************************************************
% Chapter number style: decomment if you want it with the same corpus font
% \renewcommand{\chapterNumber}{%
% \fontsize{70}{70}\usefont{\encodingdefault}{\sfdefault}{b}{n}%
% }%

% Workaround: multi-line titles
\renewcommand\formatchapter[1]{%
\begin{minipage}[b]{0.15\linewidth}
\chapterNumber
\end{minipage}%
\begin{minipage}[b]{0.70\linewidth}%length of the second row
\raggedright\spacedallcaps{#1}
\end{minipage}
}

%*********************************************************************************
% Analytical index
%*********************************************************************************
\makeindex

% Balance columns of the last page
% \let\orgtheindex\theindex
% \let\orgendtheindex\endtheindex
% \def\theindex{%
% \def\twocolumn{\begin{multicols}{2}}%
% \def\onecolumn{}%
% \clearpage
% \orgtheindex
% }
% \def\endtheindex{%
% \end{multicols}%
% \orgendtheindex
% }

% Indexing commands
\newcommand*{\keyword}[2][]{#2\index{#2@#2#1}}% as simple keyword
\newcommand{\keywordsub}[2][]{#2\index{#1!#2}}% as categorized keyword
\newcommand{\keywordpagb}[1]{\keyword[|textbf]{#1}}% as simple keyword with bold page number
\newcommand{\keywordpagi}[1]{\keyword[|textit]{#1}}% as simple keyword with italic page number
\newcommand{\keywordbold}[1]{#1\index{#1@\textbf{#1}}}% as bold keyword
\newcommand{\keyworditalic}[1]{#1\index{#1@\textsl{#1}}}% as italic keyword

%*********************************************************************************
% Impostazioni di amsmath, amssymb, amsthm
%*********************************************************************************
% Force the equation numbers to be always the same size
\makeatletter
\renewcommand{\maketag@@@}[1]{\hbox{\m@th\normalsize\normalfont#1}}%

\renewcommand{\ALG@name}{Algorithme} %% Algorithmes en Fr
\renewcommand*{\listalgorithmname}{Algorithmes}

\makeatother

% operators

\newcommand{\sgn}{\operatorname{sgn}}
\newcommand{\conv}{\operatorname{conv}}
\newcommand{\vect}{\operatorname{vect}}
\newcommand*\diff{\mathop{}\!\mathrm{d}}
\newcommand{\Var}{\mathrm{Var}}

\DeclareMathOperator*{\argmax}{arg\,max}
\DeclareMathOperator*{\argmin}{arg\,min}
\DeclareMathOperator*{\card}{card}

% comandi per gli insiemi numerici (serve il pacchetto amssymb)
\newcommand{\numberset}{\mathbb}
\newcommand{\N}{\numberset{N}}
\newcommand{\R}{\numberset{R}}

% comandi per gli insiemi
\newcommand{\set}[1]{\mathbf{#1}}
\newcommand{\setel}[1]{\mathnormal{#1}}  % o \mathrm o \mathit

% comandi per i vettori
%\newcommand{\vect}[1]{\set{#1}}
\newcommand{\vectel}[1]{\mathrm{#1}}

% comandi per i simboli matematici che denotano schemi, categorie, concetti
\newcommand{\conceptsym}[1]{\mathcal{#1}}


% reference for equation terms. use \underbrace{equation}_{\myterm{termA}}
\newcounter{term}
\renewcommand*{\theterm}{(\alph{term})}
\AtBeginDocument{%
  \let\mylabel\label
}
\newcommand{\myterm}[1]{%
  \begingroup % keep the effects of \refstepcounter local
    \refstepcounter{term}%
    \mylabel{#1}%
    \text{\theterm}%
  \endgroup
}

%*********************************************************************************
% Nuovi ambienti: definizioni, teoremi etc. etc.
%*********************************************************************************
% definizioni (serve il pacchetto amsthm)
\newtheoremstyle{classicdef}% Nome
{12pt}% Spazio che precede l’enunciato
{12pt}% Spazio che segue l’enunciato
{}% Stile del font dell’enunciato
{}% Rientro (se vuoto, non c’è rientro,
% \parindent = rientro dei capoversi)
{\scshape}% Stile del font dell’intestazione
{:}% Punteggiatura che segue l’intestazione
{.5em}% Spazio che segue l’intestazione:
% " " = normale spazio inter-parola;
% \newline = a capo
{}% Specifica l’intestazione dell’enunciato
% (normalmente viene lasciata vuota)

\newenvironment{hproof}{%
  \renewcommand{\proofname}{Idée de preuve}\proof}{\endproof}

\theoremstyle{definition}
\newtheorem{definition}{Definition}
\newtheorem{observation}{Observation}[definition]

% esempi
\theoremstyle{definition}
\newtheorem{exemple}{Exemple}[definition]

% note about definitions
\theoremstyle{remark}
\newtheorem{notabene}{Note}[definition]

% note of section
\theoremstyle{remark}
\newtheorem{note}{Note}[section]

% teoremi (serve il pacchetto amsthm)
\newtheoremstyle{classicthm}% Nome
{12pt}% Spazio che precede l’enunciato
{12pt}% Spazio che segue l’enunciato
{\itshape}% Stile del font dell’enunciato
{}% Rientro (se vuoto, non c’è rientro,
% \parindent = rientro dei capoversi)
{\scshape}% Stile del font dell’intestazione
{:}% Punteggiatura che segue l’intestazione
{.5em}% Spazio che segue l’intestazione:
% " " = normale spazio inter-parola;
% \newline = a capo
{}% Specifica l’intestazione dell’enunciato
% (normalmente viene lasciata vuota)

\theoremstyle{plain}
\newtheorem{theoreme}{Théorème}[chapter]
\newtheorem*{theoreme*}{Théorème}
\newtheorem{cor}[theoreme]{Corollaire}
\newtheorem{lem}[theoreme]{Lemme}
\newtheorem{prop}[theoreme]{Proposition}
\newtheorem{oss}[theoreme]{Observation}

%*********************************************************************************
% Impostazioni di acronym
%*********************************************************************************
\newcommand{\acroname}{Acronyme}
% \renewcommand*{\acsfont}[1]{\textssc{#1}}                 % for MinionPro
\renewcommand*{\acsfont}[1]{\textsmaller{#1}}               % customize font for long version acronyms [works only if footnote not activate]
\renewcommand*{\acffont}[1]{#1}                             % idem, but for short version of acronyms
\newcommand{\bflabel}[1]{{#1}\hfill}                      % fix the list of acronyms
\makeatletter                                               % macro that tweeks acronym package to rendere lowercase or not
\newif\if@in@acrolist
\AtBeginEnvironment{acronym}{\@in@acrolisttrue}
\newrobustcmd{\ul}[2]{\if@in@acrolist#1\else#2\fi}          % \ul{C}{c}iao defines an uppercase and lowercase variant of the same acronym
\newcommand{\ACF}[1]{{\@in@acrolisttrue\acf{#1}}}           % \ACF{<acronym>} force the defined capitalized variants of acronym letters/words
\newcommand{\ACL}[1]{{\@in@acrolisttrue\acl{#1}}}           % \ACL{<acronym}
\makeatother

%*********************************************************************************
% Impostazioni di biblatex
%*********************************************************************************
\defbibheading{bibliography}{%
\cleardoublepage
\manualmark
\phantomsection
\addcontentsline{toc}{chapter}{\tocEntry{\bibname}}
\chapter*{\bibname\markboth{\spacedlowsmallcaps{\bibname}}
{\spacedlowsmallcaps{\bibname}}}}


%*********************************************************************************
% Impostazioni di hyperref
%*********************************************************************************
\hypersetup{%
    pdfencoding=auto,
    % hyperfootnotes=false,pdfpagelabels,
    %draft,	% = elimina tutti i link (utile per stampe in bianco e nero)
    colorlinks=true, linktocpage=true, pdfstartpage=1, pdfstartview=FitV,%
    % decommenta la riga seguente per avere link in nero (per esempio per la stampa in bianco e nero)
    % colorlinks=false, linktocpage=false, pdfborder={0 0 0}, pdfstartpage=1, pdfstartview=FitV,%
    breaklinks=true, pdfpagemode=UseNone, pageanchor=true, pdfpagemode=UseOutlines,%
    plainpages=false, bookmarksnumbered, bookmarksopen=true, bookmarksopenlevel=1,%
    hypertexnames=true, pdfhighlight=/O,%nesting=true,%frenchlinks,%
    urlcolor=webbrown, linkcolor=RoyalBlue, citecolor=webgreen, %pagecolor=RoyalBlue,%
    %urlcolor=Black, linkcolor=Black, citecolor=Black, %pagecolor=Black,%
    pdftitle={\mytitle},%
    pdfauthor={\textcopyright\ \myname, \myuni, \myfaculty},%
    pdfsubject={\mysubject},%
    pdfkeywords={\mykeywords},%
    pdfcreator={LuaLaTeX},%
    pdfproducer={LaTeX with hyperref, classicthesis and arsclassica}%
}

%*********************************************************************************
% Impostazioni di graphicx
%*********************************************************************************
\graphicspath{{images/}} % cartella dove sono riposte le immagini

%*********************************************************************************
% A4 optimized margins
%*********************************************************************************
%\areaset[current]{336pt}{750pt}
%\setlength{\marginparwidth}{7em}
%\setlength{\marginparsep}{2em}%

%*********************************************************************************
% Utilities
%*********************************************************************************
% make first letter uppercase
\makeatletter
\def\upcase{\expandafter\makeupcase}
\def\makeupcase#1{\uppercase{#1}}
\makeatother

% make first letter lowercase
\makeatletter
\def\lwcase{\expandafter\makelwcase}
\def\makelwcase#1{\lowercase{#1}}
\makeatother

%*********************************************************************************
% References
%*********************************************************************************
\providecommand*{\lstlistingautorefname}{algorithme}
\providecommand*{\lstnumberautorefname}{ligne}

%*********************************************************************************
% Custom references for \autoref
%*********************************************************************************
\newcommand{\definitionautorefname}{définition}
\newcommand{\observationautorefname}{observation}
\newcommand{\noteautorefname}{note}
\newcommand{\exempleautorefname}{exemple}
\newcommand{\theoreautorefname}{théorème}

%*********************************************************************************
% Set footnote's continuos numbering and linking capabilities (using cleveref and chngcntr packages)
%*********************************************************************************
\counterwithout{footnote}{chapter}
\crefformat{footnote}{#2\footnotemark[#1]#3}

%*********************************************************************************
% Translations for package algorithmicx
%*********************************************************************************
\renewcommand{\algorithmicend}{\textbf{fin}}
\renewcommand{\algorithmicdo}{\textbf{faire}}
\renewcommand{\algorithmicwhile}{\textbf{tant que}}
\renewcommand{\algorithmicfor}{\textbf{pour}}
\renewcommand{\algorithmicreturn}{\textbf{renvoyer}}
\renewcommand{\algorithmicif}{\textbf{si}}
\renewcommand{\algorithmicthen}{\textbf{alors}}
\renewcommand{\algorithmicelse}{\textbf{sinon}}


%*********************************************************************************
% Define your commands here
%*********************************************************************************
% ...
\newcommand{\ellipse}{\dots\negthinspace}
\newcommand{\doublequotes}[1]{``#1''}
\newcommand{\ie}{i.\,e.}
\newcommand{\Ie}{I.\,e.}
\newcommand{\eg}{e.\,g.}
\newcommand{\Eg}{E.\,g.}

%*********************************************************************************
% Hypenation exceptions
%*********************************************************************************
%\hyphenation{Fortran ma-cro-istru-zio-ne nitro-idrossil-amminico}

%*********************************************************************************
% Cuboides tikz
%*********************************************************************************
\makeatletter
\def\tikz@lib@cuboid@get#1{\pgfkeysvalueof{/tikz/cuboid/#1}}

\def\tikz@lib@cuboid@setup{%
   \pgfmathsetlengthmacro{\vxx}%
      {\tikz@lib@cuboid@get{xscale}*cos(\tikz@lib@cuboid@get{xangle})*1cm}
   \pgfmathsetlengthmacro{\vxy}%
      {\tikz@lib@cuboid@get{xscale}*sin(\tikz@lib@cuboid@get{xangle})*1cm}
   \pgfmathsetlengthmacro{\vyx}%
      {\tikz@lib@cuboid@get{yscale}*cos(\tikz@lib@cuboid@get{yangle})*1cm}
   \pgfmathsetlengthmacro{\vyy}%
      {\tikz@lib@cuboid@get{yscale}*sin(\tikz@lib@cuboid@get{yangle})*1cm}
   \pgfmathsetlengthmacro{\vzx}%
      {\tikz@lib@cuboid@get{zscale}*cos(\tikz@lib@cuboid@get{zangle})*1cm}
   \pgfmathsetlengthmacro{\vzy}%
      {\tikz@lib@cuboid@get{zscale}*sin(\tikz@lib@cuboid@get{zangle})*1cm}
}

\def\tikz@lib@cuboid@draw#1--#2--#3\pgf@stop{%
    \begin{scope}[join=bevel,x={(\vxx,\vxy)},y={(\vyx,\vyy)},z={(\vzx,\vzy)}]
       % first fill the faces with global and individual style
       % then draw the grids
       \begin{scope}[canvas is yz plane at x=#1]
          \draw[cuboid/all faces,cuboid/edges,cuboid/right face] 
                (0,0) -- ++(#2,0) -- ++(0,-#3) -- ++(-#2,0) -- cycle;
          \draw[cuboid/all grids,cuboid/right grid] (0,0) grid (#2,-#3);
       \end{scope}
       \begin{scope}[canvas is xy plane at z=0]
          \draw[cuboid/all faces,cuboid/edges,cuboid/front face] 
                (0,0) -- ++(#1,0) --  ++(0,#2) -- ++(-#1,0) -- cycle;
          \draw[cuboid/all grids,cuboid/front grid] (0,0) grid (#1,#2);
       \end{scope}
       \begin{scope}[canvas is xz plane at y=#2]
          \draw[cuboid/all faces,cuboid/edges,cuboid/top face] 
                (0,0) -- ++(#1,0) --  ++(0,-#3) -- ++(-#1,0) -- cycle;
          \draw[cuboid/all grids,cuboid/top grid] (0,0) grid (#1,-#3);
       \end{scope}
       % now, draw the hidden edges
       \draw[cuboid/hidden edges] (0,#2,-#3) -- (0,0,-#3) -- (0,0,0) 
                (0,0,-#3) -- ++(#1,0,0);
       % finally, draw the visible edges 
       \begin{scope}[canvas is yz plane at x=#1]
          \draw[cuboid/all faces,cuboid/right face,cuboid/edges,fill opacity=0] 
                (0,0) -- ++(#2,0) -- ++(0,-#3) -- ++(-#2,0) -- cycle;
       \end{scope}
       \begin{scope}[canvas is xy plane at z=0]
          \draw[cuboid/all faces,cuboid/front face,cuboid/edges,fill opacity=0] 
                (0,0) -- ++(#1,0) --  ++(0,#2) -- ++(-#1,0) -- cycle;
       \end{scope}
       \begin{scope}[canvas is xz plane at y=#2]
          \draw[cuboid/all faces,cuboid/top face,cuboid/edges,fill opacity=0] 
                (0,0) -- ++(#1,0) --  ++(0,-#3) -- ++(-#1,0) -- cycle;
       \end{scope}
       % define the anchors: 8 vertices
       \path (0,#2,0) coordinate (-left top front)
                      coordinate (-left front top)
                      coordinate (-top left front)
                      coordinate (-top front left)
                      coordinate (-front top left)
                      coordinate (-front left top);
       \path (0,#2,-#3) coordinate (-left top rear)
                        coordinate (-left rear top)
                        coordinate (-top left rear)
                        coordinate (-top rear left)
                        coordinate (-rear top left)
                        coordinate (-rear left top);
       \path (0,0,-#3) coordinate (-left bottom rear)
                       coordinate (-left rear bottom)
                       coordinate (-bottom left rear)
                       coordinate (-bottom rear left)
                       coordinate (-rear bottom left)
                       coordinate (-rear left bottom);
       \path (0,0,0) coordinate (-left bottom front)
                     coordinate (-left front bottom)
                     coordinate (-bottom left front)
                     coordinate (-bottom front left)
                     coordinate (-front bottom left)
                     coordinate (-front left bottom);
       \path (#1,#2,0) coordinate (-right top front)
                       coordinate (-right front top)
                       coordinate (-top right front)
                       coordinate (-top front right)
                       coordinate (-front top right)
                       coordinate (-front right top);
       \path (#1,#2,-#3) coordinate (-right top rear)
                         coordinate (-right rear top)
                         coordinate (-top right rear)
                         coordinate (-top rear right)
                         coordinate (-rear top right)
                         coordinate (-rear right top);
       \path (#1,0,-#3) coordinate (-right bottom rear)
                        coordinate (-right rear bottom)
                        coordinate (-bottom right rear)
                        coordinate (-bottom rear right)
                        coordinate (-rear bottom right)
                        coordinate (-rear right bottom);
       \path (#1,0,0) coordinate (-right bottom front)
                      coordinate (-right front bottom)
                      coordinate (-bottom right front)
                      coordinate (-bottom front right)
                      coordinate (-front bottom right)
                      coordinate (-front right bottom);
       % centers of the 6 faces
       \coordinate (-left center) at (0,.5*#2,-.5*#3);
       \coordinate (-right center) at (#1,.5*#2,-.5*#3);
       \coordinate (-top center) at (.5*#1,#2,-.5*#3);
       \coordinate (-bottom center) at (.5*#1,0,-.5*#3);
       \coordinate (-front center) at (.5*#1,.5*#2,0);
       \coordinate (-rear center) at (.5*#1,.5*#2,-#3);
       % center of the cuboid
       \coordinate (-center) at (.5*#1,.5*#2,-.5*#3);
       % centers of the 12 edges
       \path (0,#2,-.5*#3) coordinate (-left top center) 
                           coordinate (-top left center);
       \path (.5*#1,#2,-#3) coordinate (-top rear center)
                            coordinate (-rear top center);
       \path (#1,#2,-.5*#3) coordinate (-right top center)
                            coordinate (-top right center);
       \path (.5*#1,#2,0) coordinate (-top front center)
                          coordinate (-front top center);
       \path (0,0,-.5*#3) coordinate (-left bottom center) 
                           coordinate (-bottom left center);
       \path (.5*#1,0,-#3) coordinate (-bottom rear center)
                            coordinate (-rear bottom center);
       \path (#1,0,-.5*#3) coordinate (-right bottom center)
                            coordinate (-bottom right center);
       \path (.5*#1,0,0) coordinate (-bottom front center)
                          coordinate (-front bottom center);
       \path (0,.5*#2,0) coordinate (-left front center) 
                           coordinate (-front left center);
       \path (0,.5*#2,-#3) coordinate (-left rear center)
                            coordinate (-rear left center);
       \path (#1,.5*#2,0) coordinate (-right front center)
                            coordinate (-front right center);
       \path (#1,.5*#2,-#3) coordinate (-right rear center)
                          coordinate (-rear right center);
    \end{scope}
}

\tikzset{
  pics/cuboid/.style = {
    setup code = \tikz@lib@cuboid@setup,
    background code = \tikz@lib@cuboid@draw#1\pgf@stop
  },
  pics/cuboid/.default={1--1--1},
  cuboid/.is family,
  cuboid,
  all faces/.style={fill=white},
  all grids/.style={draw=none},
  front face/.style={},
  front grid/.style={},
  right face/.style={},
  right grid/.style={},
  top face/.style={},
  top grid/.style={},
  edges/.style={},
  hidden edges/.style={draw=none},
  xangle/.initial=0,
  yangle/.initial=90,
  zangle/.initial=210,
  xscale/.initial=1,
  yscale/.initial=1,
  zscale/.initial=0.5
}

\newcommand{\tikzcuboidreset}{
\tikzset{cuboid,
  all faces/.style={fill=white},
  all grids/.style={draw=none},
  front face/.style={},
  front grid/.style={},
  right face/.style={},
  right grid/.style={},
  top face/.style={},
  top grid/.style={},
  edges/.style={},
  hidden edges/.style={draw=none},
  xangle=0,
  yangle=90,
  zangle=210,
  xscale=1,
  yscale=1,
  zscale=0.5
}
}

\newcommand{\tikzcuboidset}{\@ifstar\tikzcuboidset@star\tikzcuboidset@nostar} 
\newcommand{\tikzcuboidset@nostar}[1]{\tikzcuboidreset\tikzset{cuboid,#1}}
\newcommand{\tikzcuboidset@star}[1]{\tikzset{cuboid,#1}}
\makeatother

